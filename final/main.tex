\documentclass[12pt]{article}
\usepackage{nips13submit_e}
\usepackage{amsmath}
\usepackage{amssymb}
\usepackage{enumerate}
\usepackage{fullpage}

\nipsfinalcopy

\title{10-702: Class Project\\
Graph-valued Regression}
\author{Willie Neiswanger\\
\texttt{willie@cs.cmu.edu}
\and
Peter Schulam\\
\texttt{pschulam@cs.cmu.edu}}

\begin{document}

\maketitle

\section{Introduction}
\label{sec:introduction}

It can be difficult to estimate the parameters of a high dimensional
multivariate normal. The number of parameters is $O(p^2)$, and, if the
number of paramters grows with $n$ (the number of samples), estimation
can be especially difficult. Graphical models capture independence
assumptions in high dimensional distributions that can reduce the
number of parameters in the model. When domain knowledge can be used
to construct a graphical model with reasonable independence
assumptions, such savings can be significant. In cases when
independence assumptions are difficult to make, however, it can be
useful to estimate the structure of a graphical model from data. This
problem has been studied in the context of multivariate normal data,
where estimating the non-zero entries of the precision matrix $\Omega$
is equivalent to estimating the edges in a graph $\mathcal{G}$ where
each node $V \in \mathcal{V}$ is a random variable (dimension in the
multivariate normal).

To see why estimating the non-zero entries of $\Omega$ is equivalent
to estimating the edges present in a graphical model $\mathcal{G}$, we
can derive the conditional probability of two variables in a Gaussian
graphical model given all other variables. Suppose we have a random
vector $X \in \mathbb{R}^{10}$, and we wish to compute $P(x_1, x_2 |
x_3, \ldots, x_{10})$. Let the covariance matrix $\Sigma$ be defined
as

\begin{align}
    \Sigma &= \begin{bmatrix}
        \Sigma_{11} & \Sigma_{12} \\
        \Sigma_{21} & \Sigma_{22} \\
    \end{bmatrix}
    \text{ where }
    \Sigma_{11} = \begin{bmatrix}
        \sigma_{1} & \sigma_{12} \\
        \sigma_{12} & \sigma_{2} \\
    \end{bmatrix}
\end{align}

The conditional probability $P(x_1, x_2 | x_3, \ldots, x_{10})$ is
also normal with covariance matrix $\Sigma^\prime = (\Sigma_{11} -
\Sigma_{12} \Sigma_{22}^{-1} \Sigma_{21})^{-1} = \Omega_{11}$, where
$\Omega_{11}$ is the block of the precision matrix corresponding to
$x_1, x_2$. We then see that

\begin{align}
    p(x_1, x_2 | x_3, \ldots, x_{10}) &\propto \exp \left\{
      - \frac{1}{2}
      \begin{pmatrix}
        x_1 \\
        x_2
      \end{pmatrix}^T
      \begin{pmatrix}
        \omega_{11} & \omega_{12} \\
        \omega_{21} & \omega_{22} \\
      \end{pmatrix}
      \begin{pmatrix}
        x_1 \\
        x_1
      \end{pmatrix}
    \right\} \\
    &= \exp \left\{
      - \frac{1}{2}
      \left(
        \omega_{11} x_1^2 + 2 x_1 x_2 \omega_{12} + \omega_{22} x_2^2
      \right)
    \right\}
\end{align}

where the conditional joint distribution becomes the product of two
independent normals if $\omega_{12} = \omega_{21} = 0$. In a Gaussian
graphical model $\mathcal{G}$, the lack of an edge between two nodes
$X_i, X_j$ implies that $X_i \bot X_j | X_{-i,j}$. Since we have shown
that this is also the case when $\Omega_{ij} = \Omega_{ji} = 0$, we
can estimate the structure of a Gaussian graphical model by estimating
the non-zero entries of its precision matrix.

One of the more successful techniques for estimating a sparse
precision matrix (yielding a graphical model with few edges) is the
graphical lasso \cite{friedman2008}. The glasso is one technique for
solving the following optimization problem

\begin{align}
\label{glasso}
    \hat{\Omega} = \underset{\Omega \succ 0}{\arg\min} \left( tr(S\Omega) - \log |\Omega| + \lambda \|\Omega\|_1 \right)
\end{align}

In this paper, we study an extended version of this problem. Suppose
we have random vectors $Y$ and $X$ with dimension $p$ and $q$
respectively. \textit{Graph-valued regression} is the problem of
estimating the undirected graph $G(x)$ corresponding to conditional
independencies in the distribution $P(Y|X=x)$.

\section{Simple Graph Estimators based on Glasso}
\label{simpleEstimators}

We describe three types of graph estimators that are built upon, or
make use of, the glasso algorithm

\subsection{Parametric Estimators}

A simple approach involves assuming that $Z = (X,Y)$ is distributed
according to a multivariate Gaussian with covariance matrix

\begin{align}
    \Sigma = \left( \begin{matrix} \Sigma_X& \Sigma_{XY} \\ \Sigma_{YX} & \Sigma_Y \end{matrix} \right).
\end{align}

Under this assumption, the conditional distribution of $Y|(X=x)$ is
also distributed according to a multivariate Gaussian with covariance
matrix

\begin{align}
    \Sigma_{Y|X} = \Sigma_Y - \Sigma_{YX}\Omega_X\Sigma_{XY}.
\end{align}

We can estimate $\Sigma_X,\Sigma_Y$, and $\Sigma_{XY}$ with their
sample quantities $\hat{\Sigma}_X$,$\hat{\Sigma}_Y$, and
$\hat{\Sigma}_{XY}$, and can estimate $\Omega_X$ using the
glasso. Finally, a sparse estimate $\hat{\Omega}_{Y|X}$ of the
precision matrix $\Omega_{Y|X}$ can be achieved by plugging
$\hat{\Sigma}_{Y|X}$ into glasso.

\subsection{Kernel Smoothing Estimators}

Without making any assumptions about the marginal distribution over
$X$, one can make the assumption that the conditional distribution
$P(Y|X)$ is a multivariate Gaussian, such that

\begin{align}
    Y|(X=x) \sim N(\mu(x),\Sigma(x)).
\end{align}

Furthermore, by assuming that both $\mu(x)$ and $\Sigma(x)$ are smooth
functions of $x$, one can estimate $\Sigma(x)$ via kernel smoothing,
with the estimator

\begin{align}
    \hat{\Sigma}(x) = \sum_{i=1}^n K \left( \frac{\|x-x_i\|}{h} \right) (y_i - \hat{\mu}(x)) (y_i - \hat{\mu}(x))^{\top} / \sum_{i=1}^n K \left( \frac{\| x - x_i \|}{h} \right)
\end{align}

and

\begin{align}
    \hat{\mu}(x) = \sum_{i=1}^n K\left( \frac{\|x-x_i\|}{h} \right) y_i / \sum_{i=1}^n K\left( \frac{\|x-x_i\|}{h} \right)
\end{align}

where $K$ denotes some kernel and $h>0$ is a bandwith
parameter. Afterwards, one can apply glasso (Equation~\ref{glasso}),
using the matrix $S = \hat{\Sigma}(x)$ to estimate the graph structure
$G(x)$.

\subsection{Partition Estimators}

Both of the previous glasso-based simple graph estimators have trouble
recovering the partition of the input space
$\mathcal{X}_1,\ldots,\mathcal{X}_m$ that corresponds with the set of
output graphs. In the case of parametric estimators, we estimate a
graph that does not vary with different values of $X$, and in the case
of kernel smoothing estimators, it is computationally challenging to
reconstruct the partition.

An approach to better estimate the partition involves splitting the
input domain $\mathcal{X}$ into finitely many connected regions
$\mathcal{X}_1,\ldots,\mathcal{X}_m$, and applying glasso in each
region $\mathcal{X}_j$ to get an estimated graph
$\hat{G_j}$. Afterwards the estimated graph regression (or,
equivalently, the estimated partition) is the mapping $\hat{G}(x) =
\hat{G}_j$ for all $x \in \mathcal{X}_j$.

In order to find the parition, this paper uses ideas from the CART
(classification and regression trees) algorithm
\cite{breiman1993classification}. In particular, a tree is defined,
where nodes in the tree correspond to elements of recursively defined
partitions of a unit cube over the input space $\mathcal{X}$. The
method defined in this paper, termed Graph-optimized CART (Go-CART),
then estimates a graph for each leaf node in the tree (or
equivalently, for each element of the most-refined
partitions). Go-CART can therefore estimate the graph $G(x)$
associated with $P(Y | X=x)$ for each $x \in \mathcal{X}$.

\section{Go-CART}

Consider random vectors $X \in \mathbb{R}^d$ and $Y \in \mathbb{R}^p$
and let $(x_1,y_1),\ldots,(x_n,y_n)$ denote i.i.d samples from
$P(X,Y)$. Let $\mathcal{X}$ and $\mathcal{Y}$ denote the domains of
$X$ and $Y$. Go-CART assumes that

\begin{align}
    Y|(X=x) \sim \text{N}_p(\mu(x),\Sigma(x))
\end{align}

where $\mu:\mathbb{R}^d \rightarrow \mathbb{R}^p$ and
$\Sigma:\mathbb{R}^d \rightarrow \mathbb{R}^{p \times p}$. Go-CART is
a partition based estimator, where $\mathcal{X}$ is partitioned into
$\{ \mathcal{X}_1,\ldots,\mathcal{X}_m \}$, and glasso is applied in
each partition element $\mathcal{X}_j$ to estimate a graph
$\hat{G}_j$.

Go-CART uses a Dyadic Partitioning Tree (DPT) to partition the input
space $\mathcal{X} = [0,1]^d$, where each DPT $T$ is constructed by
recursively dividing $\mathcal{X}$ with axis-orthogonal equally-spaced
splits, producing a partition denoted $\Pi(T)$ with $m_T$ partition
elements. For an integer $N=2^K$, $K \in \mathbb{Z}_+$, let
$\mathcal{T}_N$ denote the set of DPTs such that no partition has a
side length smaller than $2^{-K}$. For a given DPT $T$, let

\begin{align}
    \mu_T(x) = \sum_{j=1}^{m_T} \mu_{\mathcal{X}_j} \cdot \mathbb{I}(x \in \mathcal{X}_j)
    \hspace{3mm} \text{ and } \hspace{3mm}
    \Omega_T(x) = \sum_{j=1}^{m_T} \Omega_{\mathcal{X}_j} \cdot \mathbb{I}(x \in \mathcal{X}_j).
\end{align}

Furthermore, let the negative conditional log-likelihood risk
$R(T,\mu_T,\Omega_T)$ and the sample risk $\hat{R}(T,\mu_T,\Omega_T)$
be

\begin{align}
    R(T,\mu_T,\Omega_T) &= \sum_{j=1}^{m_T} \mathbb{E} \left[ \left( \text{tr} \left[ \Omega_{\mathcal{X}_j} \left( (Y - \mu_{\mathcal{X}_j})(Y - \mu_{\mathcal{X}_j})^{\top} \right) \right]  - \text{log} |\Omega_{\mathcal{X}_j} | \right)  \cdot \mathbb{I}(X \in \mathcal{X}_j) \right] \\
    \hat{R}(T,\mu_T,\Omega_T) &= \frac{1}{n} \sum_{i=1}^n \sum_{j=1}^{m_T} \left[ \left( \text{tr} \left[ \Omega_{\mathcal{X}_j} \left( (y_i - \mu_{\mathcal{X}_j})(y_i - \mu_{\mathcal{X}_j})^{\top} \right) \right]  - \text{log} |\Omega_{\mathcal{X}_j} | \right)  \cdot \mathbb{I}(x_i \in \mathcal{X}_j) \right].
\end{align}

Then, we can define the \emph{penalized empirical risk minimization
  Go-CART estimator} to be

\begin{align}
    \left( \hat{T}, \{ \hat{\mu}_{\hat{\mathcal{X}}_j}, \hat{\Omega}_{\hat{\mathcal{X}}_j} \}_{j=1}^{m_{\hat{T}}} \right) 
    = \text{argmin}_{T \in \mathcal{T}_N, \mu_{\mathcal{X}_j} \in M_j, \Omega_{\mathcal{X}_j} \in \Lambda_j} 
    \left\{ \hat{R}(T,\mu_T,\Omega_T) + \text{pen}(T) \right\}
\end{align}

where

\begin{align}
    \text{pen}(T) = \gamma_n \cdot m_T \sqrt{\frac{[[T]]\text{log}2 + 2\text{log}(np)}{n}}.
\end{align}

In the above expression, the term $[[T]] > 0$ denotes a prefix code
over all DPTs $T \in \mathcal{T}_N$ such that $\sum_{T\in
  \mathcal{T}_N} 2^{-[[T]]} \leq 1$, such as

\begin{align}
    [[T]] = 3 m_T - 1 + (m_T - 1)\text{log }d / \text{log }2.
\end{align}

\section{Statistical Analysis of Go-CART}

One of the main theoretical results in this paper is a probabilistic
bound on the excess risk of the estimators obtained from both the
penalized empirical risk and the hold-out empirical risk. We refer to
these two bounds as Theorem $4.1$ and Theorem $4.2$ respectively, and
focus primarily on the proof techniques used in Theorem $4.1$ since
the proof for Theorem $4.2$ is similar, and uses many of the same
methods. The second theoretical result is a consistency result on the
estimators, showing that under the assumption that the model is
correct, the Go-CART algorithm consistently estimates the correct
partition of the input space $\mathcal{X}$.

Before discussing the theorems, we first define some notation, and
list two assumptions that are key to the results. First, define the
oracle risk $R^*$ over $\mathcal{T}_N$ as

\begin{align}
  R^* = R(T^*, \mu_T^*, \Omega_T^*)
  = \inf_{T \in \mathcal{T}_N, \mu_{\mathcal{X}_j} \in M_j, \Omega_{\mathcal{X}_j} \in \Lambda_j}
  R(T, \mu_T, \Omega_T)
\end{align}

Additionally, the authors make the following two
assumptions.

\textbf{Assumption 4.1} places a bound on the
partition-specific parameters of the model. In particular

\begin{align}
  \max_{1\le j\le m_T} \|\mu_{\mathcal{X}_j}\|_\infty \le B
\end{align}

and

\begin{align}
  &\max_{1 \le j \le M_T} \sup_{\Omega \in \Lambda_j} \log |\Omega| \le L_n \\
  &\Lambda_j = \{\Omega \in \mathbb{R}^{p \times p} :
  \Omega \text{ is positive semidefinite, symmetric and } \|\Omega\|_1 \le L_{j,n}
  \}
\end{align}

\textbf{Assumption 4.2} bounds the expected difference between two
quantities and their expectations. Let $Y = (Y_1, \ldots, Y_p)^T \in
\mathbb{R}^p$. For any $\mathcal{A} \subset \mathcal{X}$, define the
following quantities

\begin{align}
  Z_{k\ell}(\mathcal{A}) &=
  Y_k Y_\ell \mathbb{E}(X \in \mathcal{A}) - \mathbb{E}(Y_k Y_\ell \mathbb{I}(X \in \mathcal{A})) \\
  Z_{j}(\mathcal{A}) &=
  Y_j \mathbb{E}(X \in \mathcal{A}) - \mathbb{E}(Y_j \mathbb{E}(X \in \mathcal{A})
\end{align}

The authors assume that there exist constants $M_1, M_2, v_1,$ and
$v_2$ such that

\begin{align}
  \sup_{k,\ell, A} \mathbb{E} | Z_{k\ell}(\mathcal{A}|^m &\le \frac{m! M_2^{m-2} v_2}{2} \\
  \sup_{j, A} \mathbb{E} | Z_j(\mathcal{A}) |^m &\le \frac{m! M_1^{m-1} v_1}{2}
\end{align}

\subsection{Results for Penalized Empirical Risk Estimator}

Formally, Theorem $4.1$ states the following. Let $T \in
\mathcal{T}_N$ be a dyadic partitioning tree with depth at most $K$
where $K$ is some positive integer and $N = 2^K$. Let $\Pi(T) =
\{\mathcal{X}_1, \ldots, \mathcal{X}_{m_T}\}$ be the partition of
$\mathcal{X}$ induced by $T$. Finally, let $\hat{T},
\hat{\mu}_{\hat{T}}, \hat{\Omega}_{\hat{T}}$ be the estimator obtained
using the penalized empirical risk minimization Go-CART, with the
following penalty term

\begin{align}
  pen(T) = (C_1 + 1) L_n m_T \sqrt{\frac{[[T]] \log 2 + 2 \log p = \log(48/\delta)}{n}}
\end{align}

where $C_1 = 8 \sqrt{v_2} + 8 B \sqrt{v_1} + B^2$. Then Theorem $4.1$
shows that for any $\delta \in (0, 1)$ and for sufficiently large $n$
the following bound on the excess risk of the estimator holds with
probability at least $1-\delta$

\begin{align}
  R(\hat{T}, \hat{\mu}_{\hat{T}}, \hat{\Omega}_{\hat{T}}) - R^*
  \le \inf_{T \in \mathcal{T}_N} \left\{
    2 pen(T) + \inf_{\mu_{\mathcal{X}_j} \in M_j, \Omega_{\mathcal{X}_j} \in \Lambda_j}
    (R(T, \mu_T, \Omega_T) - R*)
  \right\}
\end{align}

We now discuss the proof of this result. We begin with a high-level
sketch that we will then use to guide a discussion of the specific
techniques that are used.

\textbf{Proof Sketch:} We will first establish a uniform bound over
all $T \in \mathcal{T}_N, \mu_{\mathcal{X}}, \Omega_{\mathcal{X}}$ on
the absolute value of the difference between the empirical and true
risk of $T, \mu_{\mathcal{X}}, \Omega_{\mathcal{X}}$. This gives us
the following bound that holds with probability at least $1 -
\delta/2$.

\begin{align}
  \sup_{T \in \mathcal{T}_N, \mu_j \in M_j, \Omega_j \in \Lambda_j}
  \left|
    R(T, \mu_T, \Omega_T) - \hat{R}(T, \mu_T, \Omega_T)
  \right| \le pen(T)
\end{align}

where

\begin{align}
  pen(T) = (C_1 + 1) L_n m_T \sqrt{\frac{[[T]] \log 2 + 2 \log p + \log(48/\delta)}{n}}
\end{align}

With this bound, the authors then define the following parameters for
any DPT $T$

\begin{align}
  \mu_T^0, \Omega_T^0 &= \arg\min_{\mu_T \in M_j, \Omega_T \in \Lambda_j} R(T, \mu_T, \Omega_T)
\end{align}

In words, $\mu_T^0$ and $\Omega_T^0$ are the optimal Gaussian
parameters within the partitions induced by the tree $T$ constrained
to lie in the set $M_j$ and $\Lambda_j$ for $1 \le j \le m_T$. Given
the uniform bound and this definition, the primary result follows from
the following string of inequalities with probability $1-\delta$ for
$\delta \in (0, 1)$

\begin{align}
  &R(\hat{T}, \hat{\mu}_{\hat{T}}, \hat{\Omega}_{\hat{T}}) & \\
  &\le \hat{R}(\hat{T}, \hat{\mu}_{\hat{T}}, \hat{\Omega}_{\hat{T}}) + pen(\hat{T}) & \text{w. prob } (1 - \delta/2) \\
  &= \inf_{T \in \mathcal{T}_N, \mu_{\mathcal{X}_j} \in M_j, \Omega_{\mathcal{X}_j} \in \Lambda_j}
  \left\{
    \hat{R}(T, \mu_T, \Omega_t) + pen(T)
  \right\} & \text{by def. of pen. risk est.} \\
  &\le \inf_{T \in \mathcal{T}_N}
  \left\{
    \hat{R}(T, \mu_T^0, \Omega_T^0) + pen(T)
  \right\} & \mu_T^0, \Omega_T^0 \text{ min. true risk not emp. risk} \\
  &\le \inf_{T \in \mathcal{T}_N}
  \left\{
    R(T, \mu_T^0, \Omega_T^0) + 2pen(T)
  \right\} & \text{w. prob } (1 - \delta/2) \\
  &= \inf_{T \in \mathcal{T}_N}
  \left\{
    \inf_{\mu_{\mathcal{X}_j} \in M_j, \Omega_{\mathcal{X}_j} \in \Lambda_j}
    R(T, \mu_T, \Omega_T) + 2pen(T)
  \right\} & \text{by def. of } \mu_T^0, \Omega_T^0
\end{align}

The result is obtained by subtracting the oracle risk $R^*$ from both
sides. We now fill in the details of the proof, starting with the
bound on the absolute value of the difference between the empirical
and true risk for any DPT $T$. First, define

\begin{align}
  S_{j,n} &= \frac{1}{n} \sum_{i=1}^n (y_i - \mu_{\mathcal{X}_j}) (y_i - \mu_{\mathcal{X}_j})^T \mathbb{I}(x_i \in \mathcal{X}_j) \\
  \bar{S}_j &= \mathbb{E} (Y - \mu_{\mathcal{X}_j}) (Y - \mu_{\mathcal{X}_j})^T \mathbb{I}(X \in \mathcal{X}_j)
\end{align}

Using the definition of $R$ above, we can upper bound their difference

\begin{align}
  &\left| R(T,\mu_T,\Omega_T) - \hat{R}(T, \mu_T, \Omega_T) \right| \\
  &\le \left| \sum_{j=1}^m tr[\Omega_{\mathcal{X}_j} (S_{j,n} - \hat{S}_j) \right| + 
  \left|
    \sum_{j=1}^m \log|\Omega_{\mathcal{X}_j}|
    \left[
      \frac{1}{n} \sum_{i=1}^n \mathbb{I}(x_i \in \mathcal{X}_j) - \mathbb{E I} (X \in \mathcal{X}_j)
    \right]
  \right| \\
  &\le \left| \sum_{j=1}^m \|\Omega_{\mathcal{X}_j}\|_1 \|S_{j,n} - \hat{S}_j\|_\infty \right| + 
  \sum_{j=1}^m \left| \log|\Omega_{\mathcal{X}_j}| \right|
  \left|
    \frac{1}{n} \sum_{i=1}^n \mathbb{I}(x_i \in \mathcal{X}_j) - \mathbb{E I} (X \in \mathcal{X}_j)
  \right|
\end{align}

The first inequality follows from the triangle inequality. The second
inequality follows from the following reasoning. First, we can upper
bound the trace of $\Omega_{\mathcal{X}_j} (S_{j,n} - \bar{S}_j)$ by
noting that the trace will be a sum of the diagonal of inner product
of the two matrices. This can be upper bounded by multiplying the
absolute value of each entry in $\Omega_{\mathcal{X}_j}$ by the max
entry in $S_{j,n} - \bar{S}_j$ and summing, which is equivalent to
$\|\Omega_{\mathcal{X}_j}\|_1 \|S_{j,n} - \bar{S}_j\|_\infty$. We can
upper bound the second part of the expression using the Cauchy-Schwarz
inequality.

Upper bounding the difference between the empirical and true risk with
the sum of these two terms allows us to bound each individually. We
will first bound the second expression, which we can bound uniformly
using an interesting property of the prefix codes $[[T]]$ introduced
above. Using Hoeffding's inequality, we have the following simple
bound

\begin{align}
  \mathbb{E} \left(
    \left| \frac{1}{n} \sum_{i=1}^n \mathbb{I}(x_i \in \mathcal{X}_j) - 
      \mathbb{E}(X \in \mathcal{X}_j \right| > \epsilon
  \right)
  \le 2 \exp \left\{ - 2 n \epsilon^2 \right\}
\end{align}

Now, we can achieve a uniform bound in the following way. Let
$\delta_T \in (0,1)$ be a function of $T$, then from the inquality
above we know that with probability at most $\delta_T$

\begin{align}
    \left| \frac{1}{n} \sum_{i=1}^n \mathbb{I}(x_i \in \mathcal{X}_j) - 
      \mathbb{E}(X \in \mathcal{X}_j
    \right|
    > \sqrt{\frac{\log(2/\delta_T)}{2n}}
\end{align}

Now, let $\delta_T = \delta 2^{-[[T]]}$ for each $T \in
\mathcal{T}_N$, and define $\epsilon_T =
\sqrt{\frac{\log(2/\delta_T)}{2n}}$, then

\begin{align}
  &\mathbb{P} \left(
    \sup_{T \in \mathcal{T}_N}
    \left|
      \frac{1}{n} \sum_{i=1}^n \mathbb{I}(x_i \in \mathcal{X}_j) - \mathbb{E}(X \in \mathcal{X}_j)
    \right| / \epsilon_T
    > 1
  \right) \\
  &= \mathbb{P} \left(
    \exists T \in \mathcal{T}_N
    \left|
      \frac{1}{n} \sum_{i=1}^n \mathbb{I}(x_i \in \mathcal{X}_j) - \mathbb{E}(X \in \mathcal{X}_j)
    \right|
    > \epsilon_T
  \right) \\
  &\le 2 \sum_{T \in \mathcal{T}_N} \delta_T \\
  &= 2 \sum_{T \in \mathcal{T}_N} \delta 2^{-[[T]]}
  = 2 \delta \sum_{T \in \mathcal{T}_N} 2^{-[[T]]}
  \le 2 \delta
\end{align}

Prefix codes $[[T]]$ are defined such that $\sum_{T \in \mathcal{T}_N}
2^{-[[T]]} \le 1$, giving us the final bound above. Therefore, we have
with probability at least $1 - \delta$ that

\begin{align}
  \forall T \in \mathcal{T}_N
  \left|
    \frac{1}{n} \sum_{i=1}^n \mathbb{I}(x_i \in \mathcal{X}_j) - \mathbb{E}(X \in \mathcal{X}_j)
  \right|
  \le \sqrt{\frac{[[T]] \log 2 + \log(2/\delta)}{2n}}
\end{align}

The result follows by noting that under \textbf{Assumption 4.1}
$\max_{1\le j\le m_T} \log|\Omega_{\mathcal{X}_j}| \le L_n$. In the
interest of space, we do not go into detail about the bound on
$\sum_{j=1}^m \|\Omega_{\mathcal{X}_j}\|_1 \|S_{j,n} -
\bar{S}_j\|_\infty$. The technique involves expanding $S_{j,n}$ and
$\bar{S}_j$ and grouping true expectations with their empirical
estimates. We can then use a union bound together with Bernstein's and
Hoeffding's inequality to bound the terms in the sum that we created
using the union bound.


\subsection{Tree Partition Consistency of Go-CART}

\textbf{Assumption 4.4} specifies that for adjacent regions of the true dyadic partition, either the means or the variances should be sufficiently different. Without this assumption it might be impossible to detect the boundaries of the true partition. More specifically, if $\mathcal{X}_i^0$ and $\mathcal{X}_j^0$ are adjacent partition elements of the true tree $T^*$, then we assume there exists positive constants $c_1$, $c_2$, $c_3$, and $c_4$, such that either
\begin{align}
    2 \log \left| \frac{\Sigma_{\mathcal{X}_i^0}^* + \Sigma_{\mathcal{X}_j^0}^*}{2}  \right| - \log \left| \Sigma_{\mathcal{X}_i^0}^*  \right| - \log \left| \Sigma_{\mathcal{X}_i^0}^*  \right| \geq c_4
\end{align}
or $\| \mu_{\mathcal{X}_i^0}^* - \mu_{\mathcal{X}_j^0}^* \|_2^2 \geq c_3$. Additionally, for the variances, the assumption is that the smallest eigenvalue of $\Omega_{\mathcal{X}_j^0}^*$ is s.t. $\rho_{\text{min}}(\Omega_{\mathcal{X}_j^0}^*) \geq c_1$ for all $j=1,\ldots,m_T^*$. The final assumption is that for each $T \in \mathcal{T}_N$ and each $\mathcal{A} \in \Pi(T)$, $\mathbb{P}(X\in \mathcal{A}) \geq c_2$.


Formally, \textbf{Theorem $4.3$} states that under the previous assumptions, 
\begin{align}
    \inf_{T \in \mathcal{T}_N, \Pi(T^*) \nsubseteq \Pi(T)}
    \inf_{\mu_T, \Omega_T \in \mathcal{M}_T }
    R(T,\mu_T,\Omega_T) - R(T^*,\mu_T^*,\Omega_{T^*}^*) > \min \left\{ \frac{c_1 c_2 c_3}{2}, c_2 c_4  \right\}
\end{align}
where $c_1$, $c_2$, $c_3$, and $c_4$ are given in Assumption $4.4$. Furthermore, the Go-CART estimated with both the penalized risk minimization and held-out risk minimization is \emph{tree partition consistent}, i.e. satisfies
\begin{align}
    \mathbb{P}\left( \Pi(T^*) \subset \Pi(\hat{T}) \right) \rightarrow 1 \text{ as } n\rightarrow \infty
\end{align}
where $\Pi(T^*) \subset \Pi(\hat{T})$ if tree $\Pi(\hat{T})$ can be obtained by further splitting the hyperrectangels within tree $\Pi(T^*)$ (i.e. if $\Pi(\hat{T})$ has a finer or equal partition than $\Pi(T^*)$).


\textbf{Proof Sketch:} For all $T \in \mathcal{T}_N$, $\Pi(T^*) \nsubseteq \Pi(T)$, there exists an $\mathcal{X}' \in \Pi(T)$ s.t. no $\mathcal{A} \in \Pi(T^*)$ contains $\mathcal{X}'$. Hence, there exists a minimal class of disjoint regions $\{ \mathcal{X}_1^0,\ldots,\mathcal{X}_{k'}^0 \} \in \Pi(T^*)$ s.t. $\mathcal{X}' \subset \cup_{i=1}^{k'} \mathcal{X}_i^0$, where $k' \geq 2$. We also can see that $\mathcal{X}' = \cup_{i=1}^{k'} \mathcal{X}_i^*$, where $\mathcal{X}_i^* = \mathcal{X}_i^0 \cap \mathcal{X}'$ (for $i=1,\ldots,k'$).

Define the true parameters on the regions $\mathcal{X}_1^0,\ldots,\mathcal{X}_{k'}^0$ to be $\{ \mu_{\mathcal{X}_j^*}^*, \Omega_{\mathcal{X}_j^*}^* \}_{j=1}^{k'}$. We denote the risk of $\mu_{T^*}^*$ and $\Omega_{T^*}^*$ on the region $\mathcal{X}'$ as $R(\mathcal{X}',\mu_{T^*}^*,\Omega_{T^*}^*)$, where we can obtain
\begin{align}
    R(\mathcal{X}',\mu_{T^*}^*,\Omega_{T^*}^*) 
    = \sum_{j=1}^{k'} 
    \mathbb{P}(X \in \mathcal{X}_j^*) \left[ \text{tr}(\Omega_T (\Omega_j^*)^{-1}) + \text{tr}(\Omega_T( \mu_{\mathcal{X}_j^*}^* - \mu_T )( \mu_{\mathcal{X}_j^*}^* - \mu_T )^T ) \right]
    - \mathbb{P}(X \in \mathcal{X}') \log | \Omega_T |
\end{align}
Using the bound $R(\mathcal{X}',\mu_T,\Omega_T) \geq \max\{ R(\mathcal{X}',\mu_{T^*}^*,\Omega_T), R(\mathcal{X}',\mu_T,\Omega_{T^*}^*) \}$, and proceeding by cases, we have that

\textbf{Case 1:} the $\mu$'s are different, in which case
\begin{align}
    \inf_{\mu_T, \Omega_T \in \mathcal{M}_T } R(\mathcal{X}',\mu_T,\Omega_T) - R(\mathcal{X}',\mu_{T^*}^*,\Omega_{T^*}^*) 
    \geq c_1 c_2 \inf_{\mu_T} \sum_{j=1}^{k'} \| \mu_{\mathcal{X}_j^*}^* - \mu_T \|_2^2
\end{align}
where the inequality uses the fact that $\rho_{\text{min}}(\Omega_{\mathcal{X}_j^*}^*) \geq c_1$ and $\mathbb{P}(X \in \mathcal{X}_j^*) \geq c_2$. A lower bound for the final term in this expression is achieved at $\bar{\mu}_T = \frac{1}{k'} \sum_{j=1}^{k'} \mu_{\mathcal{X}_j^*}^*$. Also, note that w.l.o.g we can assume $k'=2$ and show that
\begin{align}
    \sum_{j=1}^2 \| \mu_{\mathcal{X}_j^*}^* - \bar{\mu}_T \|_2^2 = \frac{1}{2} \| \mu_{\mathcal{X}_1^*} - \mu_{\mathcal{X}_2^*} \|_2^2 \geq \frac{c_3}{2}
\end{align}
and hence that
\begin{align}
    \inf_{\mu_T, \Omega_T \in \mathcal{M}_T } R(\mathcal{X}',\mu_T,\Omega_T) - R(\mathcal{X}',\mu_{T^*}^*,\Omega_{T^*}^*)
    \geq \frac{c_1 c_2 c_3}{2}
\end{align}


\textbf{Case 2:} the $\Omega$'s are different, in which case we have
\begin{align}
    \inf_{\mu_T, \Omega_T \in \mathcal{M}_T } R(\mathcal{X}',\mu_T,\Omega_T) - R(\mathcal{X}',\mu_{T^*}^*,\Omega_{T^*}^*) 
    \geq
    c_2 \inf_{\Sigma_T} \sum_{j=1}^{k'} \left( \text{tr}\left( \Sigma_{X_j^*}^* \Sigma_T^{-1} \right) + \log \frac{|\Sigma_T|}{|\Sigma_{X_j^*}^*|} - p \right)
\end{align}
where $\Sigma_T = \Omega_T^{-1}$. We achieve a lower bound using $\bar{\Sigma}_T = \frac{1}{2} \left( \Sigma_{\mathcal{X}_1^*} + \Sigma_{\mathcal{X}_2^*} \right)$, and plugging this into the previous expression (after assuming $k'=2$ w.l.o.g as before) gives
\begin{equation}
    \inf_{\Sigma_T} \sum_{j=1}^{2} \left( \text{tr}\left( \Sigma_{X_j^*}^* \Sigma_T^{-1} \right) + \log \frac{|\Sigma_T|}{|\Sigma_{X_j^*}^*|} - p \right)
     \geq c_4.
\end{equation}
The two statements taken together imply that 
\begin{align}
    \inf_{\mu_T, \Omega_T \in \mathcal{M}_T } R(\mathcal{X}',\mu_T,\Omega_T) - R(\mathcal{X}',\mu_{T^*}^*,\Omega_{T^*}^*) \geq c_2 c_4
\end{align}


% \subsection{Consistency of the Estimated Partition}

% The final result relies on a strong assumption that the model is
% correct. That is, that the conditional distribution of $Y$ given $X$
% is truly

% \begin{align}
%     Y | X = x \sim N_p(\mu^*_T(x), \Omega_T^*(x))
% \end{align}

% This is a rather strong assumption, but, if it holds, we can define
% the following notion of consistency. \cite{liu2010} define a tree
% estimation procedure $\hat{T}$ to be \textit{tree partition
%   consistent} if

% \begin{align}
%     P \left(
%         \Pi(T^*) \subset \Pi(\hat{T})
%     \right) \to 1
%     \text{ as } n \to \infty
% \end{align}

% where we say that a partition $\Pi(T_2) \subset \Pi(T_1)$ if $T_1$ can
% be obtained by further splitting the rectangles at the leaves of $T_2$
% (that is if $T_1$ is induces a finer partition of the space than
% $T_2$). The final result of the paper claims that both the penalized
% empirical risk and held-out risk estimators are tree partition
% consistent. We have not had time to go over this result in detail, but
% will provide additional details about the proof in future work.

\section{Conclusion}

We have reviewed the Go-CART algorithm presented in \cite{liu2010},
which builds off of results from \cite{friedman2008}. We discussed the
partition based algorithm, and its theoretical
properties. Specifically, we reviewed the techniques used in the proof
of a result that bounds the excess risk of the dyadic partitioning
tree-based graph estimator. We also reviewed the consistency proof,
which shows that, under the assumption that the DPT model is correct,
the Go-CART algorithm will consistently estimate the true partition of
the space $\mathcal{X}$.

\textbf{Contributions:} Peter focused on Sections $1$, $4.0$, and $4.1$.

\bibliographystyle{plain}
\bibliography{main}

\end{document}
