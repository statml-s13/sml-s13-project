\documentclass[12pt]{article}
\usepackage{amsmath}
\usepackage{amssymb}
\usepackage{enumerate}
\usepackage{fullpage}

\title{10-702: Project Proposal\\
Graph-valued Regression}
\author{Willie Neiswanger\\
\texttt{willie@cs.cmu.edu}
\and
Peter Schulam\\
\texttt{pschulam@cs.cmu.edu}}

\begin{document}

\maketitle

\section{Description}
\label{sec:description}

We will study methods in statistics and machine learning that focus on
regression in the space of graphs. The graphical lasso
\cite{friedman2008} estimates the precision (inverse covariance)
matrix of a random vector $\boldsymbol{Y}$ assumed to be drawn from a
multivariate Gaussian. The paper \cite{liu2009} relaxes this
assumption.

Graph-valued regression is the problem of estimating the inverse
covariance matrix of a random vector $\boldsymbol{Y}$ conditioned on
another random vector $\boldsymbol{X}$. \cite{liu2010} describes and
analyzes an approach to this problem that estimates precision matrices
within partitions of $\boldsymbol{\mathcal{X}}$ that are induced using
a variant of classification and regression trees. This method also
relies on a Gaussian assumption: $\boldsymbol{Y} | \boldsymbol{X} \sim
\mathcal{N}(\mu(x), \Sigma(x))$. We would like to explore ways in
which the non-paranormal relaxation can be applied to graph-valued
regression.

\section{Scope}
\label{sec:scope}

At minimum, we plan to read the following papers:

\begin{enumerate}[1.]
\item Graph-valued Regression \cite{liu2010}
\item The Non-paranormal: Semiparametric Estimation of High Dimensional Undirected Graphs \cite{liu2009}
\item Sparse Inverse Covariance Estimation with the Graphical Lasso \cite{friedman2008}
\item Model Selection in Gaussian Graphical Models: High-dimensional Consistency of $\ell_1$ Regularized MLE \cite{ravikumar2008}
\item Sparse Permutation Invariant Covariance Estimation \cite{rothman2008}
\item Time-varying Undirected Graphs \cite{zhou2010}
\end{enumerate}

If possible, we would like to explore the theory for a relaxation of
the Gaussian assumption used in \cite{liu2010}.

\bibliographystyle{plain}
\bibliography{proposal}

\end{document}
